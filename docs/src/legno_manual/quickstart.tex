\chapter{Legno Quickstart}

The \legno compiler (\tx{legno.py}) enables developers to compile dynamical
systems down to configurations for the analog hardware. The \legno compiler
requires that dynamical systems be specified in the \tx{dynamical system
  language}, a high level language that supports writing first-order
differential equations. The \legno compiler also accepts a specification of the
target analog device, described using the Analog Device API. The \legno compiler
generates an \tx{analog device program} which implements the target dynamical
system on the specified analog device. In the following quickstart guide, we
walk through an example where we compile a dynamical system that models the
cosine function for the \hcdc analog device. 

\subsection{Installation}

\begin{snippet}
  sudo apt-get install -y python3-pip libatlas-base-dev
\end{snippet}

The \grendel runtime is a pure python program. To install the python
dependencies, execute the following command:

\begin{snippet}
  pip3 install -f https://download.mosek.com/stable/wheel/index.html Mosek
  pip3 install -r packages.list
\end{snippet}

The \legno compiler works with a \tx{config.py}
file that specifies the relevant output directories and database files. To use
the default configuration, execute the following copy command:

\begin{snippet}
cp util/config_local.py util/config.py
\end{snippet}

\subsection{Dynamical System Program}
The following dynamical system program implements the cosine function:
\begin{dssnippet}
def make_prog():
  params = {
    'v(0)':0.0,
    'p(0)':1.0
  }
  prog = DSProg(cos)
  prog.decl_stvar('v','-p',ic='{v(0)}', params=params)
  prog.decl_stvar('p','v',ic='{p(0)}', params=params)
  prog.decl_var('pos','emit p')
  prog.range('v',-1,1)
  prog.range('p',-1,1)
  prog.max_time = 100
  prog.check()
  menv = DSSim(20)
  return prog,menv
  
\end{dssnippet}
In this dynamical system, the position \tx{pos} corresponds to the amplitude of
the cosine function. The $v$ and $p$ variables are internal variables that are
used to model the dynamics of the system. We describe each line of the program
below:

\begin{itemize}
  \item\tx{prog = DSProg('cos')}: This statement creates a new  dynamical
    system program
    with the name \tx{cos}. 
  \item \tx{prog.decl_stvar('v','-p',ic='{v(0)}',params=params)}: This statement declares a variable $v$ that has
    the dynamics $v' = -p$, and an initial value of $p(0)$, where $p(0)$ is
    defined in the parameter dictionary \tx{params}. Note that any variables in
    curly braces are replaced with values from the parameter dictionary.
  \item \tx{prog.decl_stvar('p','v',ic='{p(0)}',params=params)}: This statement declares the variable $p$ that
      has the dynamics $p = v$ and an initial value of 1.0;
  \item \tx{prog.decl_var('pos', 'emit p')}: this  statement communicates to the
    compiler that we want to measure the position of the system. We name the
    emitted signal \tx{pos}.
  \item\tx{prog.range('v',-1,1) and prog.range('p',-1,1)}: These statements
    promise to the compiler that the position and velocity variables will be
    between -1 and 1. The range of the \tx{pos} variable is inferred from the
    ranges of the \tx{p} and \tx{v} variables.
  \item\tx{prog.max_time=100}: The maximum time this simulation will ever be
    run for, in simulation units. This is useful for situations where the
    measurement hardware does not support recording signals for long periods
    of time.
  \item\tx{prog.check()}: This command checks that the program is well formed. A
    program is well formed if all the variables are bounded. 
  \item\tx{menv = DSSim(20)}: This statement defines a simulation that executes
    for 20 simulation units.  Unlike the dynamical system program, which is used
    throughout the the compilation process, the dynamical system simulation
    object is only used at the very end.
\end{itemize}

The above dynamical system program is written to \tx{cos.py} in the
\tx{bmark/bmarks/quickstart/} directory. In order for \tx{legno.py} to find the
benchmark, the \tx{bmark/bmarks/quickstart/__init__.py} file must be modified to
include the cosine benchmark in the array returned by \tx{get_benchmarks}:

\begin{pysnippet}
  import bmark.bmarks.quickstart.cos as cos

  def get_benchmarks():
     return [cos.model()]
  
\end{pysnippet}

This modification to \tx{__init__.py} enables \tx{legno.py} to find the cosine
benchmark.

\subsubsection{Executing the \tx{cos} Dynamical System with a ODE Solver}

TODO

\subsection{Compiling the \tx{cos} Program}

The \legno compiler first generates an unscaled analog device program that
implements the \tx{cos} benchmark. An analog device program (\tx{.circ})
consists of a set of block configurations and digitally programmable connections
to write to the device. This program is unscaled, meaning that the parameters
have not been changed so that dynamical system operates within the constraints
of the device. We generate the analog device program for the \tx{cos} benchmark
with the following command:

\begin{snippet}
  python3 legno.py --subset standard cos arco --abs-circuits 1
      --conc-circuits 1
\end{snippet}

This step of the compilation process is called the \lgraph compilation pass.
The \tx{--subset} flag indicates what subset of device features to use to generate
the graph. The \tx{standard} subset limits the accepted modes for each block,
which restricts the dynamic range of the device. The \tx{--abs-circuits} and
\tx{--conc-circuits} parameters control the search space explored by \lgraph
compilation pass. Specifying higher numbers to these flags produces more analog
device programs.

For the \tx{cos} dynamical system with the \tx{standard} set of features, all
unscaled circuits are stored in the following directory:
\begin{snippet}
  outputs/legno/standard/cos/abs-circ
\end{snippet}

The \lgraph command presented above produces exactly one unscaled analog device program: 

\begin{snippet}
  cos_0_0_0_0.circ
\end{snippet}

Since the analog device program is not human readable, the compiler also
produces a graph that describes the analog device program.

\begin{snippet}
  outputs/legno/standard/cos/abs-graph/cos_0_0_0_0.png
\end{snippet}

\subsubsection{Parameter Scaling with LScale}

Next, we direct the \legno compiler to scale each unscaled analog device program
in the \tx{abs-circ} directory:

\begin{snippet}
  python3 legno.py --subset standard cos jaunt --search
     --model naive --scale-circuits 1
\end{snippet}

This step of the compilation process is called the \lscale compilation pass. The
\tx{--subset} flag indicates what subset of device features to use. The
\tx{--model} argument indicates whether delta models should be considered when
scaling the system (A delta model is a model that describes the physical
hardware behavior). The \tx{naive} model assumes each block delivers its
expected behavior with no deviations. The \tx{--scale-circuits} parameter
determines how many scaled programs to produce from each unscaled program.
Finally the \tx{--search} parameter tells \legno to search for the scaling
transform that produces the best signal-to-noise ratio.

For the \tx{cos} dynamical system with the \tx{standard} set of features, the
resulting scaled programs are written to the following directory:

\begin{snippet}
  outputs/legno/standard/cos/conc-circ
\end{snippet}

The \lscale execution presented above produces exactly one scaled analog device program: 

 \begin{snippet}
  cos_0_0_0_0_s0_nq1.53d1.10b_obsslow.circ
\end{snippet}

Since the analog device program is not human readable, the compiler
also produces graphs that visually depict the circuit each program implements.
These graphs are stored in the following directory:

\begin{snippet}
  outputs/legno/standard/cos/conc-graph
\end{snippet}

\subsubsection{Source Generation}

For each scaled analog device program, legno generates a low-level \grendel
script that executes the experiment on the analog hardware. 

\begin{snippet}
  python3 legno.py --subset standard cos srcgen default --trials 1
\end{snippet}

This command generates a \grendel script for each scaled analog device program.
For the \tx{cos} dynamical system with the \tx{standard} set of features, all the \grendel scripts are
stored in the following directory:

\begin{snippet}
  outputs/legno/standard/cos/grendel
\end{snippet}

Since the \tx{cos} benchmark only has one scaled circuit, only one \grendel
script is produced:

\begin{snippet}
  cos_0_0_0_0_s0_nq1.53d1.10b_obsslow_t20_default.grendel
\end{snippet}

This grendel script can be dispatched to the \hcdc analog device with the
following commands:

\begin{snippet}
  python3 exp_driver.py scan
  python3 exp_driver.py run
\end{snippet}

These commands execute the cosine dynamical system described in
Section~\ref{XXX}. For the \tx{cos} benchmark with the \tx{standard} set of
features, the waveforms are stored in the following directory:
\begin{snippet}
  outputs/legno/standard/cos/out-waveform
\end{snippet}

The collected waveforms can then be analyzed (compared with the expected cosine function) with the following command:

\begin{snippet}
  python3 exp_driver.py analyze
\end{snippet}

The relevant visualizations are stored in the following directory:
\begin{snippet}
  outputs/legno/standard/cos/plots
\end{snippet}
