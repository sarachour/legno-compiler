
\chapter{Lookup Table (\tx{LUT} Block)}

The digital to analog converter is a digital block available on the \hcdc
device~\cite{lut.h}. Figures~\ref{lut:values} and \ref{lut:types} present the
complete list of digitally settable codes in each LUT block. The digital to
analog converter accepts one digital input and produces one digital output. The
function the LUT block implements is set by the end-user.

\noindent\textbf{Location}: Each slice on the \hcdc device contains one LUT. The
LUT may read values from ADCs resident on slice 0 (\tx{source} is
\tx{LSRC_ADC0}) or slice 2 (\tx{source} is \tx{LSRC_ADC1}) on the same tile.


\noindent\textbf{\static and \dynamic Codes}: The lookup table accepts one
\static code, \tx{source}, that determines which ADC the LUT should read from.
If \tx{source} is \tx{LSRC_ADC0}, it reads from the ADC on slice 0 of the same
tile. If \tx{source} is \tx{LSRC_ADC1}, it reads from the ADC on slice 2 of the
same tile. 

\begin{marginfigure}
  \small
  \begin{tabular}{l|l}
    code &values\\
    \hline
    source & \tx{lut_source_t}\\
  \end{tabular}
  \caption{LUT Values \cite{fu.h}}
  \label{lut:values}
\end{marginfigure}


\begin{marginfigure}
  \small
  \begin{tabular}{l|l}
    code &values\\
    \hline
    source & \static\\
  \end{tabular}
  \caption{LUT Types \cite{fu.h}}
  \label{lut:types}
\end{marginfigure}

\section{Block Function}
Given a LUT as location \hslice{chip}{tile}{slice}, the behavior of the block is
dictated by the function encoded in the lookup table. The function is
implemented as a 256-value array (\tx{MAP}), where each value is an 8-bit
number. The index into the 256 value array is the input value, and the value at
that cell is the output value. We formally describe the behavior of the block
with the following relation: 


\begin{algorithmic}
\If {\tx{source} = \tx{LSRC_ADC0}}
  \State $TABLE[\tx{adc}\hslice{chip}{tile}{0}]$
\ElsIf {\tx{source} = \tx{LSRC_ADC1}}
  \State $TABLE[\tx{adc}\hslice{chip}{tile}{2}]$
\EndIf
    
\end{algorithmic}

\subsection{\analoglib Implementation}

This function is written to the LUT block by setting each input-output pair
using the \tx{setLut} function~\cite{lut.h}. The \tx{addr} argument is the
digital code of the input, and the \tx{data} argument is the digital code of the
output.


\section{Grendel API Hook}

\grendel supports configuring the LUT block using \tx{use_lut} and
\tx{write_lut} functions. The \tx{use_lut} function sets the ADC the LUT reads
from, and the \tx{write_lut} function sets the one-input one-output function the
LUT implements:

\lstset{ 
  morekeywords={use_lut,write_lut,src},
  basicstyle=\small
}
\begin{lstlisting}
  use_lut 0 3 0 src source
  write_lut chip tile slice [input] function
\end{lstlisting}

Both commands accept a LUT location in the form of a chip, tile and slice. The
\tx{use_lut} command accepts a \tx{source} argument that determines which ADC to
read from. The \tx{source} value may be either \tx{adc0} or \tx{adc1}. The
\tx{write_lut} command accepts an function and input argument. The function
argument is a python expression and the input argument specifies the name of the
input variable that appears in the function. The provided function will be
executed on decimal values between [-1,1], and must produce values within the range [-1,1].

\subsection{Example Usage}

The following invocation configures the LUT at chip 0, tile 3, slice 0 to read
digital values from the ADC at chip 0, tile 3, slice 2. The LUT is then
configured to implement the function $0.5 sgn(in) \cdot \sqrt{in}$.

\begin{lstlisting}
  use_lut 0 3 0 src adc1
  write_lut 0 3 0 [in_] ((0.5)*math.copysign(1,in_)*(math.sqrt(abs(in_))))
\end{lstlisting}

The following invocation configures the LUT at chip 0, tile 3, slice 2 to read
digital values from the ADC at chip 0, tile 3, slice 0. The LUT is then
configured to implement the function $sin(in^2)$. 
\begin{lstlisting}
  use_lut 0 3 2 src adc0
  write_lut 0 3 2 [in_] (sin(in_*in_))
\end{lstlisting}

