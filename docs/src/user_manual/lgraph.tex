\chapter{\tx{LGraph} Compilation Pass}
\label{sec:lgraph}
The following section describes how to run the \lgraph compilation pass.

\section{Example: Synthesizing Circuits for the Cosine Function}

The \legno compiler first generates an unscaled analog device program that
implements the \tx{cos} benchmark. An analog device program (\tx{.adp})
consists of a set of block configurations and digitally programmable connections
to write to the device. This program is unscaled, meaning that the parameters
have not been changed so that dynamical system operates within the constraints
of the device. We generate the analog device program for the \tx{cos} benchmark
with the following command:

\begin{snippet}
  python3 legno.py --subset extended lgraph cos --abs-circuits 5 
      --conc-circuits 5 --max-circuits 1
\end{snippet}

This step of the compilation process is called the \lgraph compilation pass. The
\tx{--abs-circuits} and \tx{--conc-circuits} parameters control the search
space explored by \lgraph compilation pass. Specifying higher numbers to these
flags produces more analog device programs. The \tx{--subset} flag specifies
what subset of features to use. We choose the \tx{extended} subset because it
includes the broadest subset of tested features.

For the \tx{cos} dynamical system with the \tx{extended} set of features, all
unscaled circuits are stored in the following directory:

\begin{snippet}
  legno-compiler/outputs/legno/extended/cos/lgraph-adp
\end{snippet}

The \lgraph command presented above produces exactly one unscaled analog device program: 

\begin{snippet}
  cos_g0x0.adp
\end{snippet}

Since the analog device program is not human readable, the compiler also
produces a graph that describes the analog device program.

\begin{snippet}
  outputs/legno/extended/cos/lgraph-diag/cos_g0x0.png
\end{snippet}


\section{\lgraph Command Reference}

The following section is a command reference for the \lgraph command.

\begin{itemize}
\item \tx{--help}: show documentation for the command.
\item \tx{--simulate}: ignore resource constraints. This is currently not supported. 
\item \tx{--xforms}: Number of transforms to apply. A transform is a rewrite
  rule that may be applied to a differential equation. This is currently not supported. 
\item \tx{--abs-circuits}: number of abstract circuits to generate. The first
  step of graph synthesis involves generating abstract circuits.
\item \tx{--conc-circuits}: number of concrete circuits to generate per abstract
  circuits.
\item \tx{--max-circuits}: maximum number of circuits to generate.
\end{itemize}


\subsection{\tx{.adp} Naming Convention}

\lgraph generates analog device programs with the following naming convention:

\begin{snippet}
outputs/legno/<subset>/<prog>/lgraph-adp/{<prog>_g<abs>x<conc>.png
\end{snippet}

The \tx{abs} and \tx{conc} fields are the numerical identifier of the
abstract and the concrete circuit. Together, these form a unique identifier for
the circuit.


