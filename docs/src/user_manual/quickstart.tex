\chapter{Legno Quickstart}

The following quickstart walks through compiling a dynamical system program that
implements the cosine (\tx{cos}) function. This section requires you know the
model number of your board. Refer to the tag included with your board for this
information.


\section{Compiling the \tx{cos} Program}

The following section describes how to compile the cosine program. We use the
\tx{legno.py} script to compile the program. This script  is invoked multiple
times to produce a scaled analog device program (\tx{adp}). The first step of
compilation is circuit synthesis. This step assembles together a circuit of
analog blocks which implements the provided program. We will be generating one
unscaled circuit that models the cosine function with the \tx{lgraph}
compilation pass:

\begin{snippet}
  python3 legno.py lgraph --adps 1 cos
\end{snippet}

This command generates one unscaled circuit (\tx{--adps 1}) which implements
the cosine function. This circuit cannot be directly executed on the analog
device because it may contain constant values and signals which lie outside of
the operating range restrictions of the device. For this reason, we need to
scale the unscaled circuit. We will be generating ten scaled circuits that
respect the hardware restrictions with the \tx{lscale} pass:

\begin{snippet}
  python3 legno.py lscale --scale-adps 10 --calib-obj minimize_error
  --scale-method ideal --objective qty --model-number <model-number> cos
\end{snippet}

This command generates 10 scaled circuits (\tx{--scale-adps 10}) from each
unscaled circuit. Each of these scaled circuits take into account the operating
range restrictions in the device. The \tx{--objective qty} argument tells the
compiler to maximize the best-case signal-to-noise ratio on all the wires.

This invocation of \tx{lscale} assumes all blocks implement the promised
functions in the hardware specification and are not subject to manufacturing variations (\tx{--scale-method ideal}). The
resulting scaled circuit is agnostic to the underlying board and calibration
objective. Note that we still tell the compiler which board we're targeting
(\tx{--model-number <model-number>}) and how this board is calibrated
(\tx{--calib-obj minimize_error}) -- this information is needed for execution
and therefore included in the scaled circuit. Refer to Section~\ref{XXX} for
details on circuit calibration and calibration objectives.

\tx{lscale} also supports compensating for block-specific manufacturing
variations which were not eliminated during calibration. The following command
takes into account manufacturing variations when scaling the circuit to target
the board \tx{<model-number>} when it has been calibrated with the
\tx{maximize_fit} calibration objective:


\begin{snippet}
  python3 legno.py lscale --scale-adps 10 --calib-obj maximize_fit
  --scale-method phys --objective qty --model-number <model-number> cos
\end{snippet}


The above invocation generates 10 scaled circuits which take into account the
manufacturing variations present in the board (\tx{--scale-method phys}).
It is directed to compensate for the manufacturing variations which exist when
the board is calibrated with the \tx{maximize_fit} objective (\tx{--calib-obj
  maximize_fit}).

It is likely the above command will complain that it's missing empirical models
and fail. This scaling approach requires the blocks which are in use be
calibrated and characterized. It uses the characterization data to build models
which capture the behavioral deviations that it needs to compensate for. Execute
the command below to characterize any outstanding blocks.

\begin{snippet}
  python3 legno.py lcal cos --model-number <model-number>
\end{snippet}

Note that this may take several hours. Once it is finished, repeat the above
\tx{lscale} command once it is finished.

\subsection{Inspecting the Compilation Outputs}

The \tx{legno.py} script produces unscaled circuits, scaled circuits and
grendel scripts. All compilation and execution outputs for the cosine program is
stored in the following directory:

\begin{snippet}
  legno-compiler/outputs/legno/unrestricted/cos/
\end{snippet}

We call this the \textit{program directory}. The following sub-directories in the
program directory contain compilation outputs:

\begin{itemize}
\item\tx{lgraph-adp} and \tx{lgraph-diag}: These directories contain the
  unscaled analog device programs and associated diagrams. The diagrams are visual
  representations of the programs that are useful for debugging.
\item\tx{lscale-adp} and \tx{lscale-diag}: These directories contain the
  scaled analog device programs and associated diagrams. 
\item \tx{out-waveforms}: This directory contains the waveforms collected from
  the \hcdc board.
\item \tx{plots}: This directory contains visualizations of the collected
  waveforms from the \hcdc board. The \tx{wav} subdirectory contains \hcdc
  waveform visualizations. Software simulation visualizations are written
  to the \tx{sim} subdirectory. 
\item\tx{times}: this directory contains the runtime information for each
  pass.
\end{itemize}

We should see one \tx{.adp} file in the \tx{lgraph-adp}
directories, and one \tx{.gv.pdf}/\tx{.gv} file in the \tx{lgraph-diag}
directory. These files should have the following naming scheme: 

\begin{snippet}
  cos_g0.adp
\end{snippet}

The \tx{g0} identifier indicates this was the first unscaled circuit generated
by the \tx{lgraph} pass. We should see ten \tx{.adp} files in the
\tx{lscale-adp} directory and ten \tx{.gv.pdf}/\tx{.gv} files in the
\tx{lscale-diag} directory. These files should have the following naming scheme:

\begin{snippet}
  cos_g0_s1_ideal_qty_minerr_c1.adp 
\end{snippet}

The \tx{g0} identifier indicates this scaled circuit was derived from the
unscaled circuit with the \tx{g0} identifier. The \tx{s1} identifier indicates
this was the second scaled circuit generated from that particular unscaled
circuit. The \tx{qty} identifier indicates the generated circuit maximizes
signal quality. The \tx{ideal} identifier indicates the scaled circuit was generated
with the \tx{--scale-method ideal} flag. The \tx{c1} identifier indicates this
scaled circuit targets the board with model number \tx{c1}.


\section{Running the Cosine Scaled ADPs}

The \tx{lexec} subcommand in the legno compiler (\tx{legno.py}) executes
all generated scaled circuits on the analog hardware:

\begin{snippet}
python3 legno.py lexec cos --model-number <model-number>
\end{snippet}

This command may complain that it can't find the calibration codes for a
particular block. To rectify this, calibrate the chip with the following
command:


\begin{snippet}
  python3 legno.py lcal cos --model-number <model-number>
\end{snippet}

Note that this may take several hours. Once it is finished, repeat the above
\tx{lexec} command once it is finished.

If the \tx{lexec} command executed correctly, you should see a \tx{.json} file appear in
the \tx{out-waveforms} directory. This file is the waveform downloaded from the
microcontroller:

\begin{snippet}
legno-compiler/outputs/legno/unrestricted/cos/out-waveform/
\end{snippet}

\subsection{Execution with the Sigilent 1020XE Oscilloscope}:

If you are using the Sigilent 1020XE oscilloscope, include the
\tx{--osc} flag to \tx{lexec} commands to tell the runtime
from attempting to communicate with the measurement device. If you would like to
use an unsupported oscilloscope, you will need to modify the \tx{llcmd_sim.py}
file so that it configures the voltage, and time divisions on the oscilloscope, waits for
an edge trigger, and retrieves the waveforms from the oscilloscope.

If the script executed correctly, you should see a cosine waveform on your
oscilloscope.


\section{Analyzing Waveform Data}

The \tx{lwav} subcommand is used to analyze and render oscilloscope outputs. We
execute the following command to produce visualizations from the collected waveforms: 

\begin{snippet}
python3 legno.py lwav cos
\end{snippet}

This command generates plots for each measured signal and then attempts to align
the measured signal with the reference waveform. It inverts the scaling
transform (time and voltage scale coefficients) to recover the original
dynamics. The oscilloscope waveforms and visual analysis results are written to the
\tx{plots/wav} subdirectory in the program directory:

\begin{snippet}
  outputs/legno/unrestricted/cos/plots/wav
\end{snippet}


\section{Debugging Analog Device Programs}

You may observe dynamics which deviate significantly from the expected dynamics.
This may occur for a few reasons:

\begin{itemize}
\item\textbf{Incorrect Dyanamical System Program / Analog Device Program}: You
  may have mis-specified the dynamical system or found an issue with the graph
  synthesis or scaling procedure. Refer to the \textit{Troubleshooting ADPs and
    Dynamical System Programs} section for hints on how to pinpoint the issue. 

\item\textbf{Uncompensated Manufacturing Variations}: The ADP may have been
  configured to ignore manufacturing variations or insufficiently compensates
  for the manufacturing variations present in the circuit. Not all manufacturing
  variations can be compensated for in compilation. Refer to Section~\ref{XXX}
  for information on how to investigate the manufactuirng variations present in
  the hardware.

\item\textbf{Unmodelled Behavior}: Some behaviors (e.g. frequency-gain
  responses) are not characterized in the device. These behaviors are not
  captured in the hardware specification or in the empirically derived hardware
  models. Please contact the developers if you suspect there's unmodelled behavior
  -- some of these behaviors can be approximated by modifying the hardware
  specification.
\end{itemize}

\noindent~We recommend generating multiple scaled circuits and running them all then
selecting best performing circuit for further use. 

\subsection{Troubleshooting ADPs and Dynamical System Programs}

\tx{legno} supports simulating dynamical system programs and all ADPs in
software. Software simulations are useful for identifying issues with the
compiler. The following command simulates the \tx{cos} dynamical system program:

\begin{snippet}
python3 legno.py lsim cos --reference
\end{snippet}

\noindent~It produces a single plot which contains all the program variables in the
\tx{plots/sim} directory. This plot has a \tx{REF} identifier in its file name.
We can simulate the circuit emitted by \tx{lgraph} pass with the following command:

\begin{snippet}
python3 legno.py lsim cos --unscaled
\end{snippet}

\noindent~The following command simulates all the unscaled \tx{cos} ADPs emitted by
\tx{lgraph} pass and
writes a single plot (for each \tx{.adp}) containing visualizations of the trajectories to the \tx{plots/sim} directory. We
can finally simulate circuits emitted by the \tx{lscale} pass with the following command:

\begin{snippet}
python3 legno.py lsim cos
\end{snippet}

\noindent~This again will will simulate all the scaled \tx{cos} ADPs emitted by the
\tx{lscale} pass and writes a single plot (for each scaled \tx{adp}) containing
visualizations of the trajectories to the \tx{plots/sim} directory.\\


\noindent\textbf{Simulation Accuracy}: Note that the above simulation is rudimentary and does not use the board-specific
empirical models or impose operating range restrictions when executing the
circuit. It is therefore unwise to inspect circuit simulations for adps compiled
with the \tx{phys} scale method. 
