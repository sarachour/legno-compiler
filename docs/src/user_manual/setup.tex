\chapter{Installation}

The \legno compiler has been tested on OSX and Linux systems. It is a
command-line utility -- all commands in this document should be executed using the command line. 


\section{Setting up the Compilation Environment}
\subsection{Mac OS}


In order to install the dependencies for the \legno compiler, the \tx{brew} package manager (https://brew.sh/). Brew is installed using the following command:

\begin{snippet}
  /usr/bin/ruby -e \
  "\$(curl -fsSL https://raw.githubusercontent.com/Homebrew/install/master/install)"
\end{snippet}

 Once brew is installed, it may be accessed using the command \tx{brew}. Next,
 we should install the required dependences. 

\begin{snippet}
  brew install python3 git graphviz
\end{snippet}

You might need to execute the following command to get get \tx{pip3} setup:

\begin{snippet}
  brew postinstall python3 
\end{snippet}

\noindent\textbf{Testing the installation}:If this step executed correctly, each
of these commands should return the usage information for the command: 

\begin{snippet}
  pip3 --help
  python3 --help
  git --help
  dot --help
\end{snippet}

If any of these commands return the following error (where program is
\tx{pip3}, \tx{python3}, \tx{git} or \tx{dot}), please contact me for help:

\begin{snippet}
-bash: <program>: command not found
\end{snippet}


\subsection{Linux}

First, install the required dependencies

\begin{snippet}
sudo apt-get install -y python3-pip libatlas-base-dev git graphviz
\end{snippet}

\section{Setting up the Compilation Toolchain}


First, we need to download the compilation toolchain and make sure we have
selected the \tx{master} branch:

\begin{snippet}
  git clone git@github.com:sendyne/legno-compiler.git
  git checkout master
\end{snippet}

This should create a \tx{legno-compiler} directory; inside this directory is the
legno compiler toolchain.

\subsection{Setting up the Compiler}
The compiler is a pure python program, so almost all of the dependences can be
installed using the python package manager, \tx{pip}. Execute the following command
from the root directory of the \tx{legno-compiler} project to execute all the required packages:

\begin{snippet}
  pip3 install -r packages-legno.list
\end{snippet}

\section{Setting up the HCDCv2 Device}
These steps are only necessary if you wish to execute the generated programs on
the HCDCv2 device. To write the device firmware to the Arduino Due microcontroller attached
to the SA100ASY development board, you must install the Arduino IDE. To install
the Arduino IDE, download the executable from the following link and drag it do your \tx{Applications} directory:

\begin{snippet}
https://www.arduino.cc/en/Main/Software
\end{snippet}

After installing the Arduino IDE, navigate to \tx{Tools/Board: <Board Name>} in the menu, and select the \tx{Boards Manager}
option. After doing so, search for \tx{Due} and install the \tx{Arduino Sam
  Boards (32 bits ARM-Cortex M3} package. We installed \tx{1.6.12} of the
  \tx{Sam} board firmware.

You also want to install the \tx{DueTimer} package. This can be done by navigating to the 
\tx{Sketch/Include Library/Manage Libraries} option in the Arduino IDE and
searching for the \tx{DueTimer} package. We installed version \tx{1.4.7} of \tx{DueTimer}.

\subsubsection{Linux Command}

Download \tx{arduino.tar.xz} from the above arduino website and run the following to install and open the Arduino IDE:

\begin{snippet}
tar -xvf arduino-1.8.12-linux64.tar.xz 
cd arduino-1.8.12-linux64
./arduino-linux-setup.sh $USER
sudo ./install.sh
arduino
\end{snippet}

\subsection{Installing Arduino-Make}
The firmware is built using a Makefile that extends a specialized set Arduino
Makefiles. The Arduino makefiles are provided in the \tx{arduino-mk} package.

\subsubsection{MacOS Command}
The following installs the arduino makefiles on OSX:

\begin{snippet}
brew tap sudar/arduino-mk
brew install --HEAD arduino-mk
\end{snippet}

\subsubsection{Linux Command}
The following installs the arduino Makefiles on linux. Note that the package-manager sometimes has an outdated version of \tx{arduino-mk} that is missing \tx{Sam.mk}. Please refer to the troubleshooting section below if this is the case to install from source.

\begin{snippet}
sudo apt-get install -y arduino-mk
\end{snippet}

\subsection{Installing the Grendel Runtime Dependences}


The firmware communicates with a runtime which is also written in python.
Execute the following command from the root directory of the \tx{legno-compiler}
project to install all the required packages.

\begin{snippet}
  pip3 install -r packages-grendel.list
\end{snippet}

\subsection{Writing Firmware to the HCDCv2 Device}
After following these steps, it should be possible to flash the HCDCv2
firmware to the analog device. Navigate to the following directory:

\begin{snippet}
  legno-compiler/lab_bench/arduino/grendel_interp
\end{snippet}

To build the firmware, type the following command:
\begin{snippet}
  make
\end{snippet}

This should complete without incident. 


\subsubsection{[Linux] \tx{Sam.mk} not found}

Some linux package managers reference an outdated version of \tx{arduino-mk}
that does not include the \tx{Sam.mk} makefile. In these cases \tx{arduino-mk}
must be installed from source. To do so, first clone the following repository:

% git checkout e870443f4824cbb
\begin{snippet}
git clone git@github.com:sudar/Arduino-Makefile.git
sudo mkdir -p /usr/local/opt/arduino-mk/
sudo mv Arduino-Makefile/* /usr/local/opt/arduino-mk/
\end{snippet}

We tested the firmware with version \tx{1.6.0} of \tx{arduino-mk}. This version
of \tx{arduino-mk} includes version \tx{1.6.12} of the SAM libraries.

\subsubsection{[Linux] \tx{ARDUINO_DIR} not defined}

Sometimes, the arduino environment variables don't get set up properly. Add the
following to your \tx{.profile} or \tx{.bashrc} file. Execute \tx{whereis arduino} to figure out the directory to assign \tx{ARDUINO_DIR} to (it is \tx{/usr/local/bin} for me).

\begin{snippet}
export ARDUINO_DIR=/usr/local/bin/
export ARDMK_DIR=/usr/local/opt/arduino-mk/
export AVR_TOOLS_DIR=/usr/include
\end{snippet}

\subsubsection{[Linux] \tx{DueTimer} library not found}

On some linux distributions, the installed libraries are stored in a different
place than the device firmware. Execute the following command to move the
\tx{DueTimer} library to the location derived by \tx{arduino-mk} (the
\tx{ARDUINO_PLATFORM_LIB_PATH} directory).

\begin{snippet}
  mv ~/Arduino/libraries/DueTimer/ \
      ~/.arduino15/packages/arduino/hardware/sam/1.6.12/libraries/
\end{snippet}

\subsection{Writing Firmware to the Arduino Due}

Next we will upload the firmware to the
device. Please ensure the device is programmed via USB,
and that the USB cable is plugged into the programming port of the device.
This is the port closest to the DC power port. After connecting the device to
the computer, type the following command to flash the firmware to the device:

\begin{snippet}
  make upload
\end{snippet}

When you're done, navigate back to the \tx{legno-compiler} directory.

\begin{snippet}
  legno-compiler/
\end{snippet}


\subsubsection{[Linux] Permission denied: \tx{'/dev/ttyACM0'}}

This occurs on linux installations where the current user doesn't belong to the
\tx{dialout} group. Execute the following command to add yourself to the
necessary group:

\begin{snippet}
usermod -a -G dialout $USER 
\end{snippet}

After executing this command, you may need to log out and log back in again. If
the command worked successfully, you should be able to read from
\tx{/dev/ttyACM0} without getting a permission denied error. Verify this by
typing in the following command:

\begin{snippet}
cat /dev/ttyACM0
\end{snippet}

\subsubsection{[Linux] SAM-BA Operation Failed}

If you get the following error, there is issue with serial write operations.
This usually happens if the microusb connector is plugged into the wrong port
on the microcontroller. Verify that the microcontroller is properly setup
before continuing.

\subsection{Setting up the Sigilent 1020XE Oscilloscope}\label{sec:setup-osc}

This toolchain supports collecting signals with either the microcontroller or
the the Sigilent 1020XE Oscilloscope. To use the oscilloscope, make the following connections
between the oscilloscope and the board:

\begin{enumerate}
  \item Connect the first channel to pin A0 of the analog device (positive channel of
    chip[0,3,2])
  \item Connect the second channel to pin A1 of the analog device (negative channel of chip
    [0,3,2]).
  \item Connect the EXT lead to pin 25/SDA1 of the Arduino. The Arduino triggers the
    oscilloscope to record data.
  \item Connect the probe ground to the board ground.
\end{enumerate}

Next, we have to connect the oscilloscope to the network. First, connect the
oscilloscope to an Ethernet jack (the port is in the back of the device). We
also need the IP address of the device. To retrieve the IP address, press the
\tx{Utility} button on the Oscilloscope control panel. This should change the
menu at the bottom of the screen. Navigate to page 2 of the menu, select the
\tx{IO} option, and then select the \tx{IP Set} option. This should display
the IP address on the oscilloscope's screen. Write it down and check you can
reach the oscilloscope from your computer using the following command:

\begin{snippet}
ping <oscilloscope-ip>
\end{snippet}

\noindent\textbf{Trigger Pin (SDA1)}: While this pin is typically used to
trigger the oscilloscope, it can be used to trigger other external peripheral
devices (such as external signal generators) as well. 


\section{Configuring the Legno Toolchain}

The \legno toolchain must first be configured before it can be used. First, copy
the reference configuration:

\begin{snippet}
cp util/config_local.py util/config.py
\end{snippet}

The following fields may be adjusted as needed in the config file. Note that the
only field that should definitely be changed is the \tx{OSC_IP} field:

\begin{enumerate}
\item \tx{OSC_IP}: populate this field with the IP address of the oscilloscope.
  If you are not using an oscilloscope, you can leave this field as is. Refer
  to Section~\ref{sec:setup-osc} for instructions on how to get the IP address
  of the oscilloscope.
\item\tx{DEVSTATE_PATH} (default recommended): set this field to the directory where all the device
  data (calibration information, analog block models, etc) should be stored. We
  recommend leaving this field unchanged.
\item\tx{OUTPUT_PATH} (default recommended): set this field to the directory where all the compiler
  outputs should be stored. We recommend leaving this field unchanged.
\end{enumerate}

The remaining fields are automatically populated from these three fields. We
don't recommend directly setting any of these. The \tx{ARDUINO_FILE_DESC} field
points to the USB port that is linked to the Arduino and is automatically
populated by looking for the proper file descriptor in the \tx{/dev/} directory.


\section{Troubleshooting}

The following section provides fixes for common errors that occur during
installation.

\subsection{\tx{brew link} when installing \tx{python3}}

\noindent If you get the following error trying to install
\tx{python3} using brew:

\begin{snippet}
Linking /usr/local/Cellar/python/3.7.5... Error: Permission denied @ dir_s_mkdir
- /usr/local/Frameworks
\end{snippet}

\noindent Execute the following commands to fix it:

\begin{snippet}
sudo mkdir /usr/local/Frameworks
sudo chown -R \$(whoami) /usr/local/Frameworks/
brew link python
brew postinstall python3
\end{snippet}

\subsection{\tx{pyserial} missing when making firmware}


\noindent If you get the following error when trying to compile the firmware:

\begin{snippet}
  import failed, serial not found
\end{snippet}

\noindent This is likely because \tx{arduino-mk} is using a different version of python than the rest of the project. To mitigate this, install pyserial for \tx{python2}:

\begin{snippet}
pip install pyserial
\end{snippet}


\subsection{ARDUINO\_DIR not defined when making firmware}

\noindent If you get the following error while trying to compile the firmware:

\begin{snippet}
/usr/share/arduino/Arduino.mk:279: *** "ARDUINO_DIR is not defined". Stop.
\end{snippet}

\noindent This is likely because the Arduino IDE is not properly installed. Please make
sure that the Arduino IDE is in the \tx{Applications} folder in OSX. 
