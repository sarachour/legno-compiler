\chapter{Digital to Analog Converter (\tx{DAC} Block)}

The digital to analog converter is a hybrid digital-analog block available on
the \hcdc device~\cite{dac.h}. Figures~\ref{dac:values} and \ref{dac:types}
presents the complete list of digitally settable codes in each DAC block. The
digital to analog converter accepts one digital input that is either read from
memory or a lookup table, and produces one analog output current. 

\noindent\textbf{Location}: Each slice on the \hcdc device contains one
DAC. The DAC may read values from the lookup tables resident on slice 0 
(\tx{source} is \tx{DSRC_LUT0}) or slice 2 (\tx{source} is \tx{DSRC_LUT1})
of the same tile.

\noindent\textbf{\static and \dynamic Codes}: We describe the digitally settable
codes resident on the block below:

\begin{itemize}
\item\tx{enable}: Determines whether the block is enabled (true) or disabled
  (false).
\item\tx{range}: Configures the current range of the analog current emitted by
  the DAC. This range cannot be set to low mode (\tx{RANGE_LOW}).
  \item\tx{inv}: Determines whether the output of the DAC is inverted or not.
    This is useful if multiple DACs are reading from the same lookup table, as
    some DACs can by configured to negate the value of the lookup table.
  \item\tx{source}: Determines where the DAC reads the digital value from. If
    the code is set to \tx{DSRC_MEM}, the digital code stored in \tx{const_code}
    is converted to an analog signal. If the code is set to \tx{DSRC_LUT0} or
    \tx{DSRC_LUT1}, the digital code emitted by the lookup table on slices 0 and
    2 respectively is converted to an analog signal. If the DAC is configured to
    read from the lookup table, it is placed in free-running mode. This allows
    the DAC to handle dynamic digital values.  
\end{itemize}

\begin{marginfigure}
    \small
    \begin{tabular}{l|l}
      code &values\\
      \hline
      \tx{enable} &\tx{bool_t}\\
      \tx{inv}    &\tx{bool_t}\\
      \tx{range}  &\tx{range_t}\\
      \tx{source} &\tx{dac_src_t}\\
      \tx{const_code} &256\\
      \tx{pmos}\caveat&8\\
      \tx{nmos}&8\\
      \tx{gain_cal}&64\\
    \end{tabular}
    \caption{DAC Values \cite{fu.h}}
    \label{dac:values}
  \end{marginfigure}
  
  \begin{marginfigure}
    \small
    \begin{tabular}{l|l}
      code & type \\
      \hline
      \tx{enable} & \static\\
      \tx{inv}    & \static\\
      \tx{range}  &\static\\
      \tx{source} & \static\\
      \tx{const_code} & \dynamic\\
      \tx{pmos}\caveat&\hidden\\
      \tx{nmos}&\hidden\\
      \tx{gain_cal}&\hidden\\
    \end{tabular}
    \caption{DAC Code Types\cite{fu.h}}
    \label{dac:types}
  \end{marginfigure}
  
\section{Block Function}\label{dac:blockfun}

Given a DAC at location \hslice{chip}{tile}{slice}, the behavior of the block
is dictated by the relation presented below. At a high level, the DAC block converts a
digital code to an analog current. The value returned by the function is the
value of the current in $\mu A$. Any behavior not covered in the algorithm
below is undefined:

\begin{algorithmic}
  \If {\tx{enable}}
    \If {\tx{source} = \tx{DSRC_MEM}}

    \State $2 \cdot sign(\tx{inv}) \cdot scale(\tx{range}) \cdot (\tx{const_code}-128)\cdot 128^{-1}$
    \ElsIf {\tx{source} = \tx{DSRC_LUT0}}

    \State $2 \cdot sign(\tx{inv}) \cdot scale(\tx{range}) \cdot (\tx{lut}\hslice{chip}{tile}{0}-128)\cdot 128^{-1}$
    \ElsIf {\tx{source} = \tx{DSRC_LUT1}}

    \State $2 \cdot sign(\tx{inv}) \cdot scale(\tx{range}) \cdot (\tx{lut}\hslice{chip}{tile}{2}-128)\cdot 128^{-1}$
    \EndIf
 \EndIf
\end{algorithmic}

The \tx{inv} code determines whether the output signal should be inverted or
not. The \tx{range} code scales the output signal by 1x or 10x. Note
that all \static and \dynamic codes are used in the block function.

\subsection{Operating Ranges}

The magnitude of the analog output is determined by the \tx{range} code of the
DAC. The \tx{range} code is limited to medium (\tx{RANGE_MED}) or high
(\tx{RANGE_HIGH}) mode:

\begin{algorithmic}
  \State{$\tx{out} \in scale(\tx{range}) \cdot [-2 \mu A, 2 \mu A] $}
\end{algorithmic}

\subsection{\analoglib Implementation}
The \tx{dac.h} file provides a \tx{computeOutput} function that implements the
block function presented above, given a set of \dynamic and \static codes. The
returned value of this function is normalized (divided by $2 \mu A$).

\section{Calibration}
The DAC block has three hidden codes:
\begin{itemize}
\item\tx{gain_cal}: This code controls the gain of the DAC.
\item\tx{pmos} and \tx{nmos}: These codes control the magnitude of the
  \tx{gain_cal} code.
\end{itemize}
The \tx{pmos} code is always set to $0$. The remaining codes are set by the block's
calibration routine. The DAC is calibrated using the following
algorithm~\cite{dac_calib.cpp}:

\begin{algorithmic}
  \State{\tx{table} = \tx{make}()}
  \For{\tx{nmos} in $0...7$ }
  \For{\tx{gain_cal} in $0...63$ with stride 16 }
    \State {loss = obj(\tx{nmos},\tx{gain_cal})}
    \State {\tx{table} $\leftarrow$ loss,(\tx{nmos},\tx{gain_cal})}
    \EndFor
  \EndFor

  \For{\tx{gain_cal} in $0...63$}
    \State {loss = obj(\tx{table.nmos},\tx{gain_cal})}
    \State {\tx{table} $\leftarrow$ loss,(\tx{nmos},\tx{gain_cal})}
  \EndFor
  \State{return \tx{table.nmos},\tx{table.gain_cal}}
\end{algorithmic}

At a high level, the calibration algorithm iterates over \tx{nmos} and
\tx{gain_cal} codes and computes the loss for each combination of codes. The
loss function is computed using the objective function, \textit{obj}. Objective
functions are evaluated over a collection of 4
test points unless specified otherwise (\tx{const_code} that encodes
0.0,0.8,-0.8,0.5). The expected behavior is computed using the block function
specified in Section~\ref{dac:blockfun}. The DAC block supports three objective functions:
\begin{itemize}
\item\tx{CALIB_MINIMIZE_ERROR}: This objective function minimizes the average
  error between the observed signal and the expected behavior. 
\item\tx{CALIB_MAXIMIZE_DELTA_FIT}: This objective function minimizes the gain
  variance and magnitude bias of the block. The magnitude bias $b$ is computed
  by measuring the signal for test point $0.0$. For the nonzero test
  points, the gain is computed by taking the ratio of the observed to the
  expected value. The gain variance $\sigma^2$ is the computed by taking the
  variance over computed gains. The final returned loss is $min(\sigma,|b|)$.
  \item\tx{CALIB_FAST}: This objective function minimizes the error for test
    point $1.0$. This quickly calibrates the gain to have good gain characteristics.
  \end{itemize}

\noindent\textit{Assumptions}: XXX
  
\section{Profiling}

XXX profiling algorithm here XXX

\section{Grendel API Hook}

\grendel supports configuring DAC blocks using the \tx{use_dac} command. We
present the general formulation of the \tx{use_dac} command below:

\lstset{ 
  morekeywords={use_dac,src,sgn,val,rng},
  basicstyle=\small
}
\begin{lstlisting}
use_dac chip tile slice src dsrc sgn sign val value rng range
\end{lstlisting}

The \tx{use_dac} command accepts a DAC location in the form of a chip, tile and slice
index and several additional arguments (\tx{dsrc}, \tx{sign},
\tx{value} and \tx{range}) which are used set the \static and \dynamic codes in the block:
\begin{itemize}
\item\tx{dsrc} argument: This argument sets the \tx{source} \static code for
  the block, and accepts \tx{mem}, \tx{lut0} and \tx{lut1} as input values.
  These input values correspond to \tx{DSRC_MEM}, \tx{DSRC_LUT0} and \tx{DSRC_LUT1} respectively.

\item\tx{sign} argument:This argument sets the \tx{inv} \static code for the
  block, and accepts \tx{+} or \tx{-} as values, where the \tx{-} value sets the
  \tx{inv} code to true.

\item\tx{val} argument: This argument indirectly sets the \tx{const_code}
  \dynamic code of the block, and accepts a floating point value between -1 and
  1. The \tx{val} argument is converted to a digital code using the following function:
\[
  min(value \cdot 128 + 128,255)
\]
\item\tx{rng}: This argument sets the \tx{range} \static code in the block,
and accepts \tx{m} and \tx{h} as input values. The \tx{m} input value
corresponds to \tx{RANGE_MED} and the \tx{h} input value corresponds to
\tx{RANGE_HIGH}.

\end{itemize}

\subsection{Example Usage}

The following invocation configures the DAC on chip 1, tile 3, slice 0 to emit
an analog signal of $10 \mu A$:

\begin{lstlisting}
use_dac 1 3 0 src mem sgn + val 0.5 rng h
\end{lstlisting}


The following invocation configures the DAC on chip 0, tile 0, slice 0 to emit
an analog signal of $0.25 \mu A$:

\begin{lstlisting}
use_dac 0 0 0 src mem sgn + val 0.125 rng m
\end{lstlisting}



The following invocation configures the DAC on chip 0, tile 2, slice 1 to
convert the output of the LUT at chip 0, tile 2, slice 2 to an analog signal.
The resulting analog signal is scaled up by ten:

\begin{lstlisting}
use_dac 0 2 1 src lut1 sgn + val 0.0 rng h
\end{lstlisting}