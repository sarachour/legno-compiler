\chapter{Analog to Digital Converter (\tx{ADC} Block)}

The analog to digital converter is a hybrid digital-analog block available on
the \hcdc device~\cite{adc.h}. Figures~\ref{adc:values} and \ref{adc:types}
presents the complete list of digitally settable codes for each ADC block. The
analog to digital converter accepts one analog input (\tx{in}), and emits one digital
output that is then read by a lookup table.

\noindent\textbf{Location}: Each even slice on the \hcdc{device} contains one
ADC. The ADC may write values to the lookup tables resident on slice 0
 or slice 2.

\noindent\textbf{\static and \dynamic Codes}: We describe the digitally settable
codes resident on the block below:

\begin{itemize}
\item\tx{enable}: Determines whether the block is enabled (true) or disabled
  (false).
\item\tx{range}: Configures the current limit of the analog input. This code cannot be set to \tx{RANGE_LOW}
\item\tx{test} codes: Places the block in various test modes. Currently unused,
  and therefore set to false.
\end{itemize}


\begin{marginfigure}
    \small
    \begin{tabular}{l|l}
      code &values\\
      \hline
      \tx{enable} &\tx{bool_t}\\
      \tx{range} &\tx{range_t}\\
      \tx{test_en} & \tx{bool_t}\\
      \tx{test_adc} & \tx{bool_t}\\
      \tx{test_i2v} & \tx{bool_t}\\
      \tx{test_rs} & \tx{bool_t}\\
      \tx{test_rsinc} & \tx{bool_t}\\
      \tx{pmos} & 8 \\
      \tx{pmos2} & 8 \\
      \tx{nmos} & 8 \\
      \tx{i2v_cal} & 64 \\
      \tx{upper} & 64 \\
      \tx{lower} & 64 \\
      \tx{upper_fs} & 4 \\
      \tx{lower_fs} & 4 \\
     \end{tabular}
    \caption{ADC values \cite{fu.h}}
    \label{dac:values}
  \end{marginfigure}

  
\begin{marginfigure}
    \small
    \begin{tabular}{l|l}
      code &values\\
      \hline
      \tx{enable} &\static\\
      \tx{range} &\static\\
      \tx{test_en} & \static\\
      \tx{test_adc} & \static\\
      \tx{test_i2v} & \static\\
      \tx{test_rs} & \static\\
      \tx{test_rsinc} & \static\\
      \tx{pmos} & \hidden \\
      \tx{pmos2} & \hidden \\
      \tx{nmos} & \hidden \\
      \tx{i2v_cal} & \hidden \\
      \tx{upper} & \hidden \\
      \tx{lower} & \hidden \\
      \tx{upper_fs} & \hidden \\
      \tx{lower_fs} & \hidden \\
     \end{tabular}
    \caption{ADC values \cite{fu.h}}
    \label{adc:types}
  \end{marginfigure}

  
\section{Block Function}\label{adc:blockfun}

The behavior of block is dictated by the relation presented below. At a high
level, the ADC converts an analog signal to a digital code. The value returned
by the function is the digital value (0-255) emitted by the ADC. Any behavior
not covered by the algorithm below is undefined:
  
\begin{algorithmic}
  \If {\tx{enable}}
  \State {$min((scale(\tx{range})^{-1} \cdot \tx{in}) \cdot 128 + 128, 255)$}
  \EndIf
\end{algorithmic}

Since the \tx{range} code affects the current limit accepted by the ADC, it also
introduces and implicit gain into the function governing the behavior of the
block. More concretely, if \tx{range} code is set to \tx{RANGE_HIGH}, the
analog signal is scaled down by 10x (scaled by $0.1$) before being converted to
a digital signal.

\subsection{Operating Ranges}

The current range of the analog input \tx{in} is determined by the \tx{range} code of the
ADC. The \tx{range} code is limited to medium (\tx{RANGE_MED}) or high (\tx{RANGE_HIGH}) mode:

\begin{algorithmic}
  \State{$\tx{in} \in scale(\tx{range}) \cdot [-2 \mu A, 2 \mu A] $}
\end{algorithmic}


\subsection{\analoglib Implementation}

The \tx{adc.h} file provides a \tx{computeOutput} function that implements the
block function presented above, given a set of \dynamic and \static codes and an
analog input value. The analog input value is provided to the function in the
form of a normalized (divided by $2 \mu A$), floating point value. 


\section{Calibration}
The ADC block has eight hidden codes:

\begin{itemize}
\item\tx{i2v_cal}: This code controls the gain of the internal current-to-voltage
  converter in the ADC.
\item\tx{upper_fs},\tx{lower_fs},\tx{upper} and \tx{lower}: These codes affect
  how the analog signal is mapped a digital value.
\item\tx{pmos}, \tx{pmos2} and \tx{nmos}: These codes control the magnitude of the
  \tx{i2v_cal} code.
\end{itemize}

The \tx{pmos} and \tx{pmos2} codes are always set to $XXX$ and $XXX$
respectively.  The remaining six codes are set using the block's calibration
routine. The ADC is calibrated using the following
algorithm~\cite{adc_calib.cpp}.

\begin{algorithmic}
  \For{fs in 0..3}
  \State{lowerFs = fs}
  \State{upperFs = fs}
    \For{spread in 0..31}
    \For{lsign in [-1,1]}
    \For{usign in [-1,1]}
      \State{lower = $31 + spread \cdot lsign$}
      \State{upper = $31 + spread \cdot usign$}
      \For{nmos in 0..7}
        \For{i2v in 0..63 with stride=16}
          \State{score = obj(lowerFs,upperFs,lower,upper,nmos,i2v)}
          \State{tbl $\leftarrow$ score,(lowerFs,upperFs,
            lower,upper,nmos,i2v)}
        \EndFor
      \EndFor
    \EndFor
    \EndFor
    \EndFor
  \EndFor
  \State{}
  \For{i2v in 0..63}
    \State{score = obj(tbl.lowerFs,tbl.upperFs,tbl.lower,tbl.upper,tbl.nmos,i2v)}
    \State{table $\leftarrow$ score,(tbl.lowerFs,tbl.upperFs.tbl.lower,tbl.upper,tbl.nmos,i2v)}
  \EndFor
  \State{}
  \State{return tbl.lowerFs,tbl.upperFs,tbl.lower,tbl.upper,tbl.nmos,tbl.i2v}
\end{algorithmic}

At a high level, the calibration routine finds the best combination of
\tx{lowerFs}, \tx{upperFs}, \tx{lower}, \tx{nmos} and \tx{upper} values, and
then finds the best \tx{i2v_cal} code for the best combination of these codes.
The algorithm doesn't exhaustively iterate over every \tx{lower} and \tx{upper}
code value. The algorithm used to iterate over \tx{lower} and \tx{upper} codes
is lifted from the original calibration routine implemented by the hardware
designer.

\noindent\textit{Assumptions}: XXX

\section{Profiling}

XXX

\section{Grendel API Hook}

\grendel supports configuring ADC blocks using the \tx{use_adc} command. We
present the general formulation of the \tx{user_adc} command below:

\lstset{ 
  morekeywords={use_adc,rng},
  basicstyle=\small
}
\begin{lstlisting}
  use_adc chip tile slice rng range
\end{lstlisting}

The \tx{use_adc} command accepts an ADC location in the form of a chip, tile and
slice and a range argument (\tx{range}) which is used to determine the range
\static code of the ADC.  If the \tx{range} argument is set to \tx{m}, the
\tx{range} static code is set to \tx{RANGE_MED}. If the \tx{range} argument is
set to \tx{h}, the \tx{range} static code is set to \tx{RANGE_HIGH}.

\subsection{Example Usage}

This invocation configures the ADC on chip 0, tile 3, slice 2 to accept an
analog signal in $[-2,2] \mu A$.

\begin{lstlisting}
  use_adc 0 3 2 rng m 
\end{lstlisting}


This invocation configures the ADC on chip 1, tile 0, slice 1 to accept an
analog signal in $[-20,20] \mu A$.

\begin{lstlisting}
  use_adc 1 0 1 rng h
\end{lstlisting}