\chapter{Generating and Executing Grendel Scripts}
\label{sec:grendel}

\section{Example: Cosine Program}

\subsection{Anatomy of a Grendel Script}

\noindent We present the \tx{.grendel} script below generated for the cosine program
below:


\begin{snippet}
micro_reset
micro_use_osc
osc_set_volt_range 0 0.102000 1.310000
osc_set_sim_time 2.063e-03
osc_set_volt_range 1 0.102000 1.310000
osc_set_sim_time 2.063e-03
micro_set_sim_time 1.587e-03
micro_use_chip
use_integ 0 3 0 sgn + val 0.0 rng h m debug
use_integ 0 3 1 sgn + val 0.9765625 rng h m debug
use_fanout 0 3 0 0  sgn + - + rng m two
mkconn fanout 0 3 0 0 port 0 tile_output 0 3 0 0
mkconn tile_output 0 3 0 0 chip_output 0 3 2
mkconn fanout 0 3 0 0 port 1 integ 0 3 0
mkconn integ 0 3 1 fanout 0 3 0 0
mkconn integ 0 3 0 integ 0 3 1
osc_setup_trigger
micro_run
osc_get_values differential 0 1 Position outputs/legno/extended/cos/out-waveform/<file>.json
get_integ_status 0 3 0
get_integ_status 0 3 1
micro_get_status
disable integ 0 3 0
disable integ 0 3 1
disable fanout 0 3 0 0
rmconn fanout 0 3 0 0 port 0 tile_output 0 3 0 0
rmconn tile_output 0 3 0 0 chip_output 0 3 2
rmconn fanout 0 3 0 0 port 1 integ 0 3 0
rmconn integ 0 3 1 fanout 0 3 0 0
rmconn integ 0 3 0 integ 0 3 1
\end{snippet}

\noindent Next we walk through the parts of the grendel script step-by-step:

\begin{snippet}
micro_reset
micro_use_osc
\end{snippet}

\noindent the \tx{micro_reset} resets the
state of the micro-controller (not the analog device). The \tx{micro_use_osc}
command tells the microcontroller to emit a trigger signal on the \tx{SDA1} pin
right before it begins executing the program:

\begin{snippet}
osc_set_volt_range 0 0.102000 1.310000
osc_set_sim_time 2.063e-03
osc_set_volt_range 1 0.102000 1.310000
osc_set_sim_time 2.063e-03
\end{snippet}

\noindent the \tx{osc_set_sim_time} command sends the amount of time the
oscilloscope should record for. This is automatically set by the compiler to be
slightly more than the scaled simulation time. The \tx{osc_set_volt_range} is
the voltage range the oscilloscope should monitor. These quantities are
converted to voltage and time divisions by the oscilloscope driver.

\begin{snippet}
micro_set_sim_time 1.587e-03
micro_use_chip
\end{snippet}

\noindent the \tx{micro_set_sim_time} command tells the microcontroller how long
to wait for after triggering the simulation. The \tx{micro_use_chip} command
tells the microcontroller that the analog chip will be used by the program.

\begin{snippet}
use_integ 0 3 0 sgn + val 0.0 rng h m debug
use_integ 0 3 1 sgn + val 0.9765625 rng h m debug
use_fanout 0 3 0 0  sgn + - + rng m two
\end{snippet}

\noindent the \tx{use_integ} and \tx{use_fanout} blocks configure the
blocks in the circuit. The first \tx{use_integ} command sets the input of the
integrator to high (\tx{h}) and the output of the integrator to medium (\tx{m}).
The \tx{debug} flag tells the integrator to watch out for overflows.
The output is noninverted (\tx{sgn +}) and the initial condition is 0.0. The
fanout is set to be in medium mode (\tx{m}). The second current copy is negated
(\tx{+ - +}). 

\noindent\textbf{Use Commands and Calibration/Profiling}: The calibration and
profiling routines find all the use commands in the script and generate
calibration or profiling commands for each use command. Both calibration and
profiling are performed for each block configuration. 

\begin{snippet}
mkconn fanout 0 3 0 0 port 0 tile_output 0 3 0 0
mkconn tile_output 0 3 0 0 chip_output 0 3 2
mkconn fanout 0 3 0 0 port 1 integ 0 3 0
mkconn integ 0 3 1 fanout 0 3 0 0
mkconn integ 0 3 0 integ 0 3 1
\end{snippet}

\noindent the \tx{mkconn} command programs the required connections for
integrating the circuit.

\begin{snippet}
osc_setup_trigger
micro_run
osc_get_values differential 0 1 Position outputs/legno/extended/cos/out-waveform/<file>.json
get_integ_status 0 3 0
get_integ_status 0 3 1
micro_get_status
\end{snippet}

\noindent the \tx{osc_setup_trigger} command sets up the trigger on the
oscilloscope. The \tx{micro_run} command runs the program. The
\tx{osc_get_values} command retrieves the data from the oscilloscope. The
waveform is computed by taking the differential signal between channel 0 and 1,
and is named \tx{Position}. For brevity we omit the complete path to the
\tx{.json} file. Finally, the script retrieves the overflow status of the
integrators and the status of the microcontroller.

\noindent\textbf{The Output Waveform}: All of the post-processing and analysis
routines process the output waveform produced by the \tx{osc_get_values}
command. The existence of this output waveform is used to determine if the
script has been executed. 

\begin{snippet}
disable integ 0 3 0
disable integ 0 3 1
disable fanout 0 3 0 0
rmconn fanout 0 3 0 0 port 0 tile_output 0 3 0 0
rmconn tile_output 0 3 0 0 chip_output 0 3 2
rmconn fanout 0 3 0 0 port 1 integ 0 3 0
rmconn integ 0 3 1 fanout 0 3 0 0
rmconn integ 0 3 0 integ 0 3 1
\end{snippet}

\noindent The \tx{disable} and \tx{rmconn} teardown the program circuit by
removing programmed connections and disabling used blocks.

\subsection{Generating Grendel Scripts with \tx{Legno}}

For each scaled analog device program, legno generates a low-level \grendel
script that executes the experiment on the analog hardware. 

\begin{snippet}
  python3 legno.py --subset extended srcgen cos --hwenv osc --trials 1
\end{snippet}

This command generates a \grendel script for each scaled analog device program.
The \tx{--hwenv} parameter specifies the surrounding hardware environment
(e.g. oscilloscope). Since we're using the Sigilent 1020XE oscilloscope, we use
\tx{--hwenv osc}, which contains the channel configuration described in section~\ref{setup-osc}.

\noindent\textbf{no oscilloscope}: If you're not using a supported measurement
device and want to discard the voltage and time information, use
\tx{--hwenv noosc} -- this will still generate a trigger signal, but will not
generate the commands to setup the measurement device. Note these commands can
always be ignored during execution using the \tx{--no-oscilloscope} flag. 

For the \tx{cos} dynamical system with the \tx{extended} set of features, all
the \grendel scripts are stored in the following directory:

\begin{snippet}
  outputs/legno/extended/cos/grendel
\end{snippet}

Since the \tx{cos} benchmark only has one scaled circuit, only one \grendel
script is produced:

\begin{snippet}
  cos_g0x0_s0_ngd3.00a3.77v1.77c97.00_obsfast_t20_osc.grendel
\end{snippet}

We will refer to this script as \tx{<file>.grendel} for the remainder of this
example.

\subsection{Calibrating Blocks in the Grendel Script}

The grendel runtime (\tx{grendel.py}) executes \tx{.grendel} scripts generated
by the \legno compiler. Before executing the script, we must calibrate any
uncalibrated blocks. The \tx{--calib-obj} argument specifies which calibration
objective to use (\tx{min_error} or \tx{max_fit}).


\begin{snippet}
python3 grendel.py calibrate --calib-obj max_fit \
  outputs/legno/extended/cos/grendel/<file>.grendel 
\end{snippet}

\noindent The calibration information is stored in \tx{device-state/state.db}, and reused
for subsequent executions. Any blocks that already have calibration information
in \tx{state.db} are automatically skipped. If you wish to recalibrate these
blocks, use the \tx{--recompute} flag. 


\noindent\textbf{No Oscilloscope}: If you are not using the
Sigilent 1020XE oscilloscope, include the \tx{--no-oscilloscope} flag to prevent
the runtime from attempting to communicate with the measurement device.

\subsection{Executing the Grendel Script}


We can run the benchmark after all the block have been calibrated. Executing the
following command to executes the program on the analog device:

\begin{snippet}
  python3 grendel.py run --calib-obj max_fit \
  outputs/legno/extended/cos/grendel/<file>.grendel
\end{snippet}

\noindent This should write the configuration to the device, and execute the cosine
function for 20 simulation units.

\noindent\textbf{No Oscilloscope}: If you are not using the
Sigilent 1020XE oscilloscope, include the \tx{--no-oscilloscope} flag to prevent
the runtime from attempting to communicate with the measurement device.

\begin{comment}
\section{Mass-Execution and Analysis (oscilloscope required)}
The \tx{exp_driver.py} script automatically runs and analyzes \tx{.grendel}
scripts in the \tx{output} directory. The \tx{exp_driver.py} script works by
invoking \tx{grendel.py} and populates the following subdirectories in the program
directory (Section XXX).

\begin{itemize}
\item\tx{out-waveform}[OSC]: this directory contains the output waveforms
  collected by the oscilloscope during execution.
\item\tx{plots}[OSC]: this directory contains visualizations of the output
  waveforms collected by the oscilloscope. These plots are generated by
  \tx{exp_driver.py}.
\end{itemize}

We first scan for all \tx{.grendel} files in the \tx{output directory}:

\begin{snippet}
  python3 exp_driver.py scan
\end{snippet}

Next, we mass-execute any outstanding \tx{.grendel} scripts,
calibrating blocks when necessary. This mass execution step ignores any grendel
scripts that already have output waveforms (it scans the grendel scripts for a
\tx{osc_get_values} command and checks to see if the associated file exists). 

\begin{snippet}
  python3 exp_driver.py run --calibrate
\end{snippet}
\end{comment}